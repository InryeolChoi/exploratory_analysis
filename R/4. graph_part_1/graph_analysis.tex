% Options for packages loaded elsewhere
\PassOptionsToPackage{unicode}{hyperref}
\PassOptionsToPackage{hyphens}{url}
%
\documentclass[
]{article}
\usepackage{amsmath,amssymb}
\usepackage{iftex}
\ifPDFTeX
  \usepackage[T1]{fontenc}
  \usepackage[utf8]{inputenc}
  \usepackage{textcomp} % provide euro and other symbols
\else % if luatex or xetex
  \usepackage{unicode-math} % this also loads fontspec
  \defaultfontfeatures{Scale=MatchLowercase}
  \defaultfontfeatures[\rmfamily]{Ligatures=TeX,Scale=1}
\fi
\usepackage{lmodern}
\ifPDFTeX\else
  % xetex/luatex font selection
\fi
% Use upquote if available, for straight quotes in verbatim environments
\IfFileExists{upquote.sty}{\usepackage{upquote}}{}
\IfFileExists{microtype.sty}{% use microtype if available
  \usepackage[]{microtype}
  \UseMicrotypeSet[protrusion]{basicmath} % disable protrusion for tt fonts
}{}
\makeatletter
\@ifundefined{KOMAClassName}{% if non-KOMA class
  \IfFileExists{parskip.sty}{%
    \usepackage{parskip}
  }{% else
    \setlength{\parindent}{0pt}
    \setlength{\parskip}{6pt plus 2pt minus 1pt}}
}{% if KOMA class
  \KOMAoptions{parskip=half}}
\makeatother
\usepackage{xcolor}
\usepackage[margin=1in]{geometry}
\usepackage{color}
\usepackage{fancyvrb}
\newcommand{\VerbBar}{|}
\newcommand{\VERB}{\Verb[commandchars=\\\{\}]}
\DefineVerbatimEnvironment{Highlighting}{Verbatim}{commandchars=\\\{\}}
% Add ',fontsize=\small' for more characters per line
\usepackage{framed}
\definecolor{shadecolor}{RGB}{248,248,248}
\newenvironment{Shaded}{\begin{snugshade}}{\end{snugshade}}
\newcommand{\AlertTok}[1]{\textcolor[rgb]{0.94,0.16,0.16}{#1}}
\newcommand{\AnnotationTok}[1]{\textcolor[rgb]{0.56,0.35,0.01}{\textbf{\textit{#1}}}}
\newcommand{\AttributeTok}[1]{\textcolor[rgb]{0.13,0.29,0.53}{#1}}
\newcommand{\BaseNTok}[1]{\textcolor[rgb]{0.00,0.00,0.81}{#1}}
\newcommand{\BuiltInTok}[1]{#1}
\newcommand{\CharTok}[1]{\textcolor[rgb]{0.31,0.60,0.02}{#1}}
\newcommand{\CommentTok}[1]{\textcolor[rgb]{0.56,0.35,0.01}{\textit{#1}}}
\newcommand{\CommentVarTok}[1]{\textcolor[rgb]{0.56,0.35,0.01}{\textbf{\textit{#1}}}}
\newcommand{\ConstantTok}[1]{\textcolor[rgb]{0.56,0.35,0.01}{#1}}
\newcommand{\ControlFlowTok}[1]{\textcolor[rgb]{0.13,0.29,0.53}{\textbf{#1}}}
\newcommand{\DataTypeTok}[1]{\textcolor[rgb]{0.13,0.29,0.53}{#1}}
\newcommand{\DecValTok}[1]{\textcolor[rgb]{0.00,0.00,0.81}{#1}}
\newcommand{\DocumentationTok}[1]{\textcolor[rgb]{0.56,0.35,0.01}{\textbf{\textit{#1}}}}
\newcommand{\ErrorTok}[1]{\textcolor[rgb]{0.64,0.00,0.00}{\textbf{#1}}}
\newcommand{\ExtensionTok}[1]{#1}
\newcommand{\FloatTok}[1]{\textcolor[rgb]{0.00,0.00,0.81}{#1}}
\newcommand{\FunctionTok}[1]{\textcolor[rgb]{0.13,0.29,0.53}{\textbf{#1}}}
\newcommand{\ImportTok}[1]{#1}
\newcommand{\InformationTok}[1]{\textcolor[rgb]{0.56,0.35,0.01}{\textbf{\textit{#1}}}}
\newcommand{\KeywordTok}[1]{\textcolor[rgb]{0.13,0.29,0.53}{\textbf{#1}}}
\newcommand{\NormalTok}[1]{#1}
\newcommand{\OperatorTok}[1]{\textcolor[rgb]{0.81,0.36,0.00}{\textbf{#1}}}
\newcommand{\OtherTok}[1]{\textcolor[rgb]{0.56,0.35,0.01}{#1}}
\newcommand{\PreprocessorTok}[1]{\textcolor[rgb]{0.56,0.35,0.01}{\textit{#1}}}
\newcommand{\RegionMarkerTok}[1]{#1}
\newcommand{\SpecialCharTok}[1]{\textcolor[rgb]{0.81,0.36,0.00}{\textbf{#1}}}
\newcommand{\SpecialStringTok}[1]{\textcolor[rgb]{0.31,0.60,0.02}{#1}}
\newcommand{\StringTok}[1]{\textcolor[rgb]{0.31,0.60,0.02}{#1}}
\newcommand{\VariableTok}[1]{\textcolor[rgb]{0.00,0.00,0.00}{#1}}
\newcommand{\VerbatimStringTok}[1]{\textcolor[rgb]{0.31,0.60,0.02}{#1}}
\newcommand{\WarningTok}[1]{\textcolor[rgb]{0.56,0.35,0.01}{\textbf{\textit{#1}}}}
\usepackage{graphicx}
\makeatletter
\def\maxwidth{\ifdim\Gin@nat@width>\linewidth\linewidth\else\Gin@nat@width\fi}
\def\maxheight{\ifdim\Gin@nat@height>\textheight\textheight\else\Gin@nat@height\fi}
\makeatother
% Scale images if necessary, so that they will not overflow the page
% margins by default, and it is still possible to overwrite the defaults
% using explicit options in \includegraphics[width, height, ...]{}
\setkeys{Gin}{width=\maxwidth,height=\maxheight,keepaspectratio}
% Set default figure placement to htbp
\makeatletter
\def\fps@figure{htbp}
\makeatother
\setlength{\emergencystretch}{3em} % prevent overfull lines
\providecommand{\tightlist}{%
  \setlength{\itemsep}{0pt}\setlength{\parskip}{0pt}}
\setcounter{secnumdepth}{-\maxdimen} % remove section numbering
\ifLuaTeX
  \usepackage{selnolig}  % disable illegal ligatures
\fi
\usepackage{bookmark}
\IfFileExists{xurl.sty}{\usepackage{xurl}}{} % add URL line breaks if available
\urlstyle{same}
\hypersetup{
  pdftitle={graph\_analysis},
  pdfauthor={inryeol},
  hidelinks,
  pdfcreator={LaTeX via pandoc}}

\title{graph\_analysis}
\author{inryeol}
\date{2024-08-05}

\begin{document}
\maketitle

\begin{Shaded}
\begin{Highlighting}[]
\FunctionTok{rm}\NormalTok{(}\AttributeTok{list=}\FunctionTok{ls}\NormalTok{())}
\end{Highlighting}
\end{Shaded}

\section{변수 가져오기}\label{uxbcc0uxc218-uxac00uxc838uxc624uxae30}

\begin{enumerate}
\def\labelenumi{\arabic{enumi}.}
\item
  변수 이름 : pollution
\item
  변수들의 타입 지정하기 (순서대로)

  \begin{itemize}
  \tightlist
  \item
    numeric, character, factor, numeric, numeric
  \end{itemize}

\begin{Shaded}
\begin{Highlighting}[]
\NormalTok{class }\OtherTok{=} \FunctionTok{c}\NormalTok{(}\StringTok{"numeric"}\NormalTok{, }\StringTok{"character"}\NormalTok{, }\StringTok{"factor"}\NormalTok{, }\StringTok{"numeric"}\NormalTok{, }\StringTok{"numeric"}\NormalTok{)}
\NormalTok{pollution }\OtherTok{=} \FunctionTok{read.csv}\NormalTok{(}\StringTok{"../../dataset/avgpm25.csv"}\NormalTok{, }\AttributeTok{colClasses =}\NormalTok{ class)}
\end{Highlighting}
\end{Shaded}
\end{enumerate}

\begin{quote}
이후 구조를 확인한다.
\end{quote}

\begin{Shaded}
\begin{Highlighting}[]
\FunctionTok{head}\NormalTok{(pollution)}
\end{Highlighting}
\end{Shaded}

\begin{verbatim}
##        pm25  fips region longitude latitude
## 1  9.771185 01003   east -87.74826 30.59278
## 2  9.993817 01027   east -85.84286 33.26581
## 3 10.688618 01033   east -87.72596 34.73148
## 4 11.337424 01049   east -85.79892 34.45913
## 5 12.119764 01055   east -86.03212 34.01860
## 6 10.827805 01069   east -85.35039 31.18973
\end{verbatim}

\begin{Shaded}
\begin{Highlighting}[]
\FunctionTok{str}\NormalTok{(pollution)}
\end{Highlighting}
\end{Shaded}

\begin{verbatim}
## 'data.frame':    576 obs. of  5 variables:
##  $ pm25     : num  9.77 9.99 10.69 11.34 12.12 ...
##  $ fips     : chr  "01003" "01027" "01033" "01049" ...
##  $ region   : Factor w/ 2 levels "east","west": 1 1 1 1 1 1 1 1 1 1 ...
##  $ longitude: num  -87.7 -85.8 -87.7 -85.8 -86 ...
##  $ latitude : num  30.6 33.3 34.7 34.5 34 ...
\end{verbatim}

\section{1차원 그래프}\label{uxcc28uxc6d0-uxadf8uxb798uxd504}

\subsection{요약통계량}\label{uxc694uxc57duxd1b5uxacc4uxb7c9}

\begin{Shaded}
\begin{Highlighting}[]
\FunctionTok{summary}\NormalTok{(pollution}\SpecialCharTok{$}\NormalTok{pm25)}
\end{Highlighting}
\end{Shaded}

\begin{verbatim}
##    Min. 1st Qu.  Median    Mean 3rd Qu.    Max. 
##   3.383   8.549  10.047   9.836  11.356  18.441
\end{verbatim}

결과

\begin{itemize}
\item
  크게 기울지는 않았다.
\item
  상당수가 12를 넘고 있다.
\end{itemize}

\subsection{boxplot}\label{boxplot}

\begin{quote}
pollution 중 pm25의 Boxplot을 보자.
\end{quote}

\begin{Shaded}
\begin{Highlighting}[]
\FunctionTok{boxplot}\NormalTok{(pollution}\SpecialCharTok{$}\NormalTok{pm25, }\AttributeTok{col=}\StringTok{"blue"}\NormalTok{)}
\end{Highlighting}
\end{Shaded}

\includegraphics{graph_analysis_files/figure-latex/unnamed-chunk-5-1.pdf}

\begin{itemize}
\tightlist
\item
  여기서 5와 15를 콧수염(whiskers)이라고 한다.
\item
  IQR = 사분위간 범위 = (3사분위수) - (1사분위수) = (전체 자료의 중간값)
\item
  whiskers를 벗어나는 경우를 이상치(outlier)로 볼 수 있나?

  \begin{itemize}
  \tightlist
  \item
    꼭 그렇지는 않다. 다만 평균에서 먼 자료 정도로 보면 된다.
  \end{itemize}
\end{itemize}

\begin{quote}
pm25가 15보다 큰 케이스를 살펴보자.
\end{quote}

\begin{Shaded}
\begin{Highlighting}[]
\CommentTok{\# no.1}
\FunctionTok{with}\NormalTok{(pollution, }\FunctionTok{head}\NormalTok{(pm25[pm25 }\SpecialCharTok{\textgreater{}} \DecValTok{15}\NormalTok{]))}
\end{Highlighting}
\end{Shaded}

\begin{verbatim}
## [1] 16.19452 15.80378 18.44073 16.66180 15.01573 17.42905
\end{verbatim}

\begin{Shaded}
\begin{Highlighting}[]
\CommentTok{\# no.2}
\FunctionTok{subset}\NormalTok{(pollution, pm25 }\SpecialCharTok{\textgreater{}} \DecValTok{15}\NormalTok{)}
\end{Highlighting}
\end{Shaded}

\begin{verbatim}
##        pm25  fips region longitude latitude
## 46 16.19452 06019   west -119.9035 36.63837
## 50 15.80378 06029   west -118.6833 35.29602
## 51 18.44073 06031   west -119.8113 36.15514
## 53 16.66180 06037   west -118.2342 34.08851
## 55 15.01573 06047   west -120.6741 37.24578
## 61 17.42905 06065   west -116.8036 33.78331
## 76 16.25190 06099   west -120.9588 37.61380
## 78 16.18358 06107   west -119.1661 36.23465
\end{verbatim}

\begin{Shaded}
\begin{Highlighting}[]
\CommentTok{\# no.3}
\FunctionTok{head}\NormalTok{(pollution[pollution}\SpecialCharTok{$}\NormalTok{pm25 }\SpecialCharTok{\textgreater{}} \DecValTok{15}\NormalTok{,], }\DecValTok{8}\NormalTok{)}
\end{Highlighting}
\end{Shaded}

\begin{verbatim}
##        pm25  fips region longitude latitude
## 46 16.19452 06019   west -119.9035 36.63837
## 50 15.80378 06029   west -118.6833 35.29602
## 51 18.44073 06031   west -119.8113 36.15514
## 53 16.66180 06037   west -118.2342 34.08851
## 55 15.01573 06047   west -120.6741 37.24578
## 61 17.42905 06065   west -116.8036 33.78331
## 76 16.25190 06099   west -120.9588 37.61380
## 78 16.18358 06107   west -119.1661 36.23465
\end{verbatim}

\begin{itemize}
\tightlist
\item
  지역은 전부 서부라고 나온다.
\item
  west이면서 fips 코드 앞자리가 06인 곳 : 캘리포니아
\end{itemize}

\subsection{Histogram}\label{histogram}

pm25의 히스토그램을 한번 보자.

breaks=100으로 해서 잘게 짜르고, 밀집도를 볼 수 있게 한다.

\begin{Shaded}
\begin{Highlighting}[]
\CommentTok{\# hist of PM2.5}
\FunctionTok{hist}\NormalTok{(pollution}\SpecialCharTok{$}\NormalTok{pm25, }\AttributeTok{col=}\StringTok{"green"}\NormalTok{, }\AttributeTok{breaks=}\DecValTok{100}\NormalTok{)}
\FunctionTok{rug}\NormalTok{(pollution}\SpecialCharTok{$}\NormalTok{pm25)}
\end{Highlighting}
\end{Shaded}

\includegraphics{graph_analysis_files/figure-latex/unnamed-chunk-7-1.pdf}

\subsection{저수준 그래픽}\label{uxc800uxc218uxc900-uxadf8uxb798uxd53d}

그래프 위에 무언가를 씌울 때 사용한다.

주로 선을 그려 중위수나 기준을 표기한다.

\subsubsection{예시 1}\label{uxc608uxc2dc-1}

pm25의 histogram. 기준선1을 10에, 기준선2를 12에 그려준다.

\begin{Shaded}
\begin{Highlighting}[]
\CommentTok{\# hist of PM2.5}
\FunctionTok{hist}\NormalTok{(pollution}\SpecialCharTok{$}\NormalTok{pm25, }\AttributeTok{col=}\StringTok{"green"}\NormalTok{)}
\FunctionTok{abline}\NormalTok{(}\AttributeTok{v =} \DecValTok{12}\NormalTok{, }\AttributeTok{lwd=}\DecValTok{2}\NormalTok{) }\CommentTok{\# 기준치}
\FunctionTok{abline}\NormalTok{(}\AttributeTok{v =} \FunctionTok{median}\NormalTok{(pollution}\SpecialCharTok{$}\NormalTok{pm25), }\AttributeTok{col=}\StringTok{"blue"}\NormalTok{, }\AttributeTok{lwd=}\DecValTok{4}\NormalTok{) }\CommentTok{\# }
\end{Highlighting}
\end{Shaded}

\includegraphics{graph_analysis_files/figure-latex/unnamed-chunk-8-1.pdf}

\subsubsection{예시 2}\label{uxc608uxc2dc-2}

pm25의 boxplot. 기준선을 파란색으로, y=12에 그려준다.

\begin{Shaded}
\begin{Highlighting}[]
\CommentTok{\# boxplot + abline of pm25}
\FunctionTok{boxplot}\NormalTok{(pollution}\SpecialCharTok{$}\NormalTok{pm25)}
\FunctionTok{abline}\NormalTok{(}\AttributeTok{h =} \DecValTok{12}\NormalTok{) }\CommentTok{\# 기준치}
\end{Highlighting}
\end{Shaded}

\includegraphics{graph_analysis_files/figure-latex/unnamed-chunk-9-1.pdf}

\subsection{Barplot}\label{barplot}

범주형 자료(categorical data)를 정리할 때 좋다.

먼저 table을 써서 정리를 한 다음, barplot을 그려준다.

\begin{Shaded}
\begin{Highlighting}[]
\FunctionTok{barplot}\NormalTok{(}\FunctionTok{table}\NormalTok{(pollution}\SpecialCharTok{$}\NormalTok{region), }\AttributeTok{col=}\StringTok{"wheat"}\NormalTok{)}
\end{Highlighting}
\end{Shaded}

\includegraphics{graph_analysis_files/figure-latex/unnamed-chunk-10-1.pdf}

\textbf{west와 east의 순서를 바꾸고 싶다면?}

\begin{itemize}
\tightlist
\item
  factor를 다시 지정해야 한다.
\end{itemize}

\begin{Shaded}
\begin{Highlighting}[]
\NormalTok{pollution}\SpecialCharTok{$}\NormalTok{region }\OtherTok{=} \FunctionTok{factor}\NormalTok{(pollution}\SpecialCharTok{$}\NormalTok{region, }\AttributeTok{levels =} \FunctionTok{c}\NormalTok{(}\StringTok{"west"}\NormalTok{, }\StringTok{"east"}\NormalTok{))}
\FunctionTok{barplot}\NormalTok{(}\FunctionTok{table}\NormalTok{(pollution}\SpecialCharTok{$}\NormalTok{region), }\AttributeTok{col=}\StringTok{"purple"}\NormalTok{)}
\end{Highlighting}
\end{Shaded}

\includegraphics{graph_analysis_files/figure-latex/unnamed-chunk-11-1.pdf}

\section{2차원 그래픽}\label{uxcc28uxc6d0-uxadf8uxb798uxd53d}

\subsection{다차원 boxplot}\label{uxb2e4uxcc28uxc6d0-boxplot}

\begin{quote}
지역별 pm2.5을 도출해보자.
\end{quote}

\begin{Shaded}
\begin{Highlighting}[]
\FunctionTok{boxplot}\NormalTok{(pm25}\SpecialCharTok{\textasciitilde{}}\NormalTok{region, }\AttributeTok{data =}\NormalTok{ pollution, }\AttributeTok{col=}\StringTok{"red"}\NormalTok{)}
\end{Highlighting}
\end{Shaded}

\includegraphics{graph_analysis_files/figure-latex/unnamed-chunk-12-1.pdf}

\subsection{다차원 histogram}\label{uxb2e4uxcc28uxc6d0-histogram}

\begin{quote}
지역별 pm2.5의 수치를 구해보자.
\end{quote}

\begin{Shaded}
\begin{Highlighting}[]
\FunctionTok{par}\NormalTok{(}\AttributeTok{mfrow=}\FunctionTok{c}\NormalTok{(}\DecValTok{2}\NormalTok{,}\DecValTok{1}\NormalTok{))}
\FunctionTok{hist}\NormalTok{(pollution[pollution}\SpecialCharTok{$}\NormalTok{region }\SpecialCharTok{==} \StringTok{"west"}\NormalTok{,]}\SpecialCharTok{$}\NormalTok{pm25,}
     \AttributeTok{col=}\StringTok{"green"}\NormalTok{)}
\FunctionTok{hist}\NormalTok{(pollution[pollution}\SpecialCharTok{$}\NormalTok{region }\SpecialCharTok{==} \StringTok{"west"}\NormalTok{,]}\SpecialCharTok{$}\NormalTok{pm25,}
     \AttributeTok{col=}\StringTok{"green"}\NormalTok{)}
\end{Highlighting}
\end{Shaded}

\includegraphics{graph_analysis_files/figure-latex/unnamed-chunk-13-1.pdf}

\subsection{Scatterplot}\label{scatterplot}

\begin{quote}
latitude에 따른 pm2.5를 구해보자.
\end{quote}

\begin{Shaded}
\begin{Highlighting}[]
\FunctionTok{plot}\NormalTok{(pollution}\SpecialCharTok{$}\NormalTok{pm25}\SpecialCharTok{\textasciitilde{}}\NormalTok{pollution}\SpecialCharTok{$}\NormalTok{latitude,}
     \AttributeTok{xlab=}\StringTok{"latitude"}\NormalTok{, }\AttributeTok{ylab=}\StringTok{"pm25"}\NormalTok{)}
\FunctionTok{abline}\NormalTok{(}\AttributeTok{h=}\DecValTok{12}\NormalTok{, }\AttributeTok{lty=}\DecValTok{2}\NormalTok{)}
\end{Highlighting}
\end{Shaded}

\includegraphics{graph_analysis_files/figure-latex/unnamed-chunk-14-1.pdf}

\begin{quote}
\textbf{latitude에 따른 pm2.5. 색으로 region 구분해주기}
\end{quote}

\begin{Shaded}
\begin{Highlighting}[]
\FunctionTok{plot}\NormalTok{(pollution}\SpecialCharTok{$}\NormalTok{pm25}\SpecialCharTok{\textasciitilde{}}\NormalTok{pollution}\SpecialCharTok{$}\NormalTok{latitude,}
     \AttributeTok{xlab=}\StringTok{"latitude"}\NormalTok{, }\AttributeTok{ylab=}\StringTok{"pm25"}\NormalTok{,}
     \AttributeTok{col=}\NormalTok{pollution}\SpecialCharTok{$}\NormalTok{region)}
\FunctionTok{abline}\NormalTok{(}\AttributeTok{h=}\DecValTok{12}\NormalTok{, }\AttributeTok{lty=}\DecValTok{2}\NormalTok{)}
\end{Highlighting}
\end{Shaded}

\includegraphics{graph_analysis_files/figure-latex/unnamed-chunk-15-1.pdf}

\begin{quote}
지역에 따른 구분. latitude별 pm25
\end{quote}

\begin{Shaded}
\begin{Highlighting}[]
\FunctionTok{par}\NormalTok{(}\AttributeTok{mfrow=}\FunctionTok{c}\NormalTok{(}\DecValTok{1}\NormalTok{,}\DecValTok{2}\NormalTok{), }\AttributeTok{mar=}\FunctionTok{c}\NormalTok{(}\DecValTok{5}\NormalTok{, }\DecValTok{4}\NormalTok{, }\DecValTok{2}\NormalTok{, }\DecValTok{1}\NormalTok{))}
\FunctionTok{with}\NormalTok{(}\FunctionTok{subset}\NormalTok{(pollution, region }\SpecialCharTok{==} \StringTok{"West"}\NormalTok{),}
     \FunctionTok{plot}\NormalTok{(pollution}\SpecialCharTok{$}\NormalTok{pm25}\SpecialCharTok{\textasciitilde{}}\NormalTok{pollution}\SpecialCharTok{$}\NormalTok{latitude, }
          \AttributeTok{main=}\StringTok{"West"}\NormalTok{, }\AttributeTok{ylab=}\StringTok{"pm25"}\NormalTok{, }
          \AttributeTok{xlab=}\StringTok{"latitude"}\NormalTok{))}
\FunctionTok{with}\NormalTok{(}\FunctionTok{subset}\NormalTok{(pollution, region }\SpecialCharTok{==} \StringTok{"East"}\NormalTok{),}
     \FunctionTok{plot}\NormalTok{(pollution}\SpecialCharTok{$}\NormalTok{pm25}\SpecialCharTok{\textasciitilde{}}\NormalTok{pollution}\SpecialCharTok{$}\NormalTok{latitude, }
          \AttributeTok{main=}\StringTok{"East"}\NormalTok{, }\AttributeTok{ylab=}\StringTok{"pm25"}\NormalTok{, }
          \AttributeTok{xlab=}\StringTok{"latitude"}\NormalTok{))}
\end{Highlighting}
\end{Shaded}

\includegraphics{graph_analysis_files/figure-latex/unnamed-chunk-16-1.pdf}

\begin{Shaded}
\begin{Highlighting}[]
\FunctionTok{par}\NormalTok{(}\AttributeTok{mfrow=}\FunctionTok{c}\NormalTok{(}\DecValTok{1}\NormalTok{,}\DecValTok{2}\NormalTok{), }\AttributeTok{mar=}\FunctionTok{c}\NormalTok{(}\DecValTok{5}\NormalTok{, }\DecValTok{4}\NormalTok{, }\DecValTok{2}\NormalTok{, }\DecValTok{1}\NormalTok{))}
\FunctionTok{plot}\NormalTok{(pm25}\SpecialCharTok{\textasciitilde{}}\NormalTok{latitude,}\AttributeTok{data=}\NormalTok{pollution[pollution}\SpecialCharTok{$}\NormalTok{region}\SpecialCharTok{==}\StringTok{"west"}\NormalTok{,],}\AttributeTok{main=}\StringTok{"West"}\NormalTok{)}
\FunctionTok{plot}\NormalTok{(pm25}\SpecialCharTok{\textasciitilde{}}\NormalTok{latitude,}\AttributeTok{data=}\NormalTok{pollution[pollution}\SpecialCharTok{$}\NormalTok{region}\SpecialCharTok{==}\StringTok{"east"}\NormalTok{,],}\AttributeTok{main=}\StringTok{"East"}\NormalTok{)}
\end{Highlighting}
\end{Shaded}

\includegraphics{graph_analysis_files/figure-latex/unnamed-chunk-17-1.pdf}

\textbf{tip : 여백을 지정할때 mar()을 사용한다.}

\section{그래픽용 도구}\label{uxadf8uxb798uxd53duxc6a9-uxb3c4uxad6c}

여러 창을 한번에 띄울때는 다음과 같은 함수를 사용한다.

\begin{itemize}
\item
  \texttt{quartz()} : 맥에서 사용
\item
  \texttt{windows()} : 윈도우에서 사용
\item
  \texttt{x11()} : 리눅스에서 사용
\end{itemize}

또한 pdf로 저장할 때는 다음과 같은 절차를 따른다.

\begin{enumerate}
\def\labelenumi{\arabic{enumi}.}
\item
  \texttt{pdf()} 함수로 열어준다.
\item
  그래프를 편집한다.
\item
  \texttt{dev.off()}로 닫아준다. (닫아야 그래프 편집내용이 저장.)
\end{enumerate}

\end{document}
